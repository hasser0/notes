\documentclass[../main.tex]{subfiles}

\begin{document}
\subsection{Binary Logistic regression}
Binary logistic regression is a discriminative model for classification that suppose a conditional distribution
\begin{equation*}
    \p(y=c|\vec{x},\vec{\theta}) \sim \text{Bern}(\sigma(\vec{w}^T \vec{x}))
\end{equation*}
Let $p=\p(y=1|\vec{x},\vec{\theta})$, then
\begin{align*}
    p &= \sigma(\vec{w}^T \vec{x})\\
    &= \dfrac{1}{1 + e^{-\vec{w}^T \vec{x}}}\\
    \Longleftrightarrow 
    p + pe^{-\vec{w}^T \vec{x}} &= 1\\
    \Longleftrightarrow 
    e^{-\vec{w}^T \vec{x}} &= \dfrac{1 - p}{p}\\
    \Longleftrightarrow 
    \vec{w}^T \vec{x} &= \log\dfrac{p}{1 - p}    
\end{align*}
This means that logistic regression is equivalent to linear regression of log odds ratio. Usually, vanilla logistic regression is trained using iterative methods maximizing the likelihood, this is
\begin{align*}
    \vec{w} &= \argmax_{\vec{w}} \mathcal{L}(\vec{w})\\
    &= \argmin_{\vec{w}} \text{NLL}(\vec{w})\\
    &= \argmin_{\vec{w}} -\sum_{n=1}^N y_n\log\sigma(\vec{w}^T\vec{x}_n) +  (1-y_n)\log\Big[1-\sigma(\vec{w}^T\vec{x}_n)\Big]
\end{align*}
Deriving the $\text{NLL}(\vec{w})$ w.r.t. $\vec{w}$ we get the gradient
\begin{align*}
    \nabla_{\vec{w}} \text{NLL} 
    &= -\sum_{n=1}^N 
    \dfrac{y_n}{\sigma(\vec{w}^T\vec{x}_n)}
    \dfrac{\partial\sigma(\vec{w}^T\vec{x}_n)}{\partial\vec{w}^T\vec{x}_n}
    \dfrac{\partial\vec{w}^T\vec{x}_n}{\partial\vec{w}} 
    +
    \dfrac{1-y_n}{1-\sigma(\vec{w}^T\vec{x}_n)}
    \dfrac{\partial\Big[1-\sigma(\vec{w}^T\vec{x}_n)\Big]}{\partial\vec{w}^T\vec{x}_n} \dfrac{\partial\vec{w}^T\vec{x}_n}{\partial\vec{w}}\\
    &= -\sum_{n=1}^N 
    \dfrac{y_n}{\sigma(\vec{w}^T\vec{x}_n)}
    \sigma(\vec{w}^T\vec{x}_n)\Big[1-\sigma(\vec{w}^T\vec{x}_n)\Big]\vec{x}_n
    -
    \dfrac{1-y_n}{1-\sigma(\vec{w}^T\vec{x}_n)}
    \sigma(\vec{w}^T\vec{x}_n)\Big[1-\sigma(\vec{w}^T\vec{x}_n)\Big]\vec{x}_n\\
    &= -\sum_{n=1}^N 
    y_n\Big[1-\sigma(\vec{w}^T\vec{x}_n)\Big]\vec{x}_n
    -
    (1-y_n)
    \sigma(\vec{w}^T\vec{x}_n)\vec{x}_n\\
    &= \sum_{n=1}^N 
    \sigma(\vec{w}^T\vec{x}_n)\vec{x}_n
    -y_n\vec{x}_n\\
    &= \sum_{n=1}^N 
    \Big[\sigma(\vec{w}^T\vec{x}_n)-y_n\Big]\vec{x}_n\\
\end{align*}
and also we can get the Hessian 
\begin{align*}
    \nabla_{\vec{w}} \nabla_{\vec{w}} \text{NLL}
    &= \sum_{n=1}^N 
    \nabla_{\vec{w}} \Big[\sigma(\vec{w}^T\vec{x}_n)-y_n\Big]\vec{x}_n\\
    &= \sum_{n=1}^N
    \sigma(\vec{w}^T\vec{x}_n)\Big[1-\sigma(\vec{w}^T\vec{x}_n)\Big]\vec{x}_n\vec{x}_n^T\\
    &= \mat{X}^T \text{diag}\Big(\mu_1-\mu_1^2,\dots,\mu_n-\mu_n^2)\Big) \mat{X}
\end{align*}
where $\mat{X}$ is the entire dataset and $\mu_i=\sigma(\vec{w}^T\vec{x}_i)$. With these components we can find the MLE parameters using stochastic gradient descent
\begin{align*}
    \vec{w}_{t+1} &\leftarrow \vec{w}_{t} - \eta_t\vec{g}_t \\
    &\Longleftrightarrow\\
    \vec{w}_{t+1} &\leftarrow \vec{w}_{t} - \eta_t\Big[\sigma(\vec{w}^T\vec{x}_n)-y_n\Big]\vec{x}_n
\end{align*}
The perceptron algorithm is a limit case of logistic regression where sigmoid function $\sigma(\cdot)$ is replace by a heavistep function $\mathbb{I}(\cdot)$, the algorithm stay the same, but updating is not applied for correct predictions.
Alternatively we can use second order Newton's method, also called iteratively reweighted least squares for logistic regression.
\begin{align*}
    \vec{w}_{t+1} &\leftarrow \vec{w}_{t} - \eta_t\mat{H}^{-1}\vec{g} \\
\end{align*}
In this case $\mat{H}$ is the already calculated Hessian and $\vec{g}$ is the gradient.
\subsection{Multinomial logistic regression}
Multinomial logistic regression is the case for multiple classes, the difference is the conditional distribution assumption
\begin{equation*}
    \p(y=c|\vec{x},\vec{\theta}) \sim \text{Cat}(\mathcal{S}(\mat{W} \vec{x}))
\end{equation*}
All calculations are kind of similar to binary case for gradient and for Hessian are out of the bounds of this notes.
\end{document}