\documentclass[../main.tex]{subfiles}

\begin{document}

\subsection{Basis}
Decision theory is a mathematical framework for taking decisions where randomness is present. In general there are four components in every decision made using decision theory; \textbf{actions}, \textbf{states}, \textbf{distributions} and \textbf{loss functions}.

Actions can be define as a set of values to be chosen, states are generally an unknown variable usually generated from a distribution where each state has an associated probability. Finally loss is define for every pair of possible interactions between actions and states; given an action or decision $\delta$ and a state $s$ we incur a loss 
\begin{equation*}
    \lo
\end{equation*}
\end{document}
