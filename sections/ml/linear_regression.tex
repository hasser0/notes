% !TEX root = ../../main.tex
\documentclass[../../main.tex]{subfiles}

\begin{document}
Linear regression is a model for regression task, which suppose the dependent variable is gaussian
\begin{equation*}
    \p(y|\vec{x},\vec{\theta}) = \normal(\vec{w}^T\vec{x},\sigma)
\end{equation*}
for the case homocedastic regression. Heterocedastic regression suppose the variance is a function of the input features. The parameters of the model are usually calculated using MLE and ordinary least squares methods, yielding the same results both. 
\subsection{Ordinary least squares}
For the case of ordinary least squares, we want to minimize the residual sum of squares
\begin{align*}
    \text{RSS}(\vec{w})
    &=\frac{1}{2}\Big(\mat{X}\vec{w}-\vec{y}\Big)^T \Big(\mat{X}\vec{w}-\vec{y}\Big)\\
    &=\frac{1}{2}\Big(
        \vec{w}^T\mat{X}^T\mat{X}\vec{w}+
        -2\vec{y}^T\mat{X}\vec{w}-\vec{y}^T\vec{y}^T
    \Big)\\
\end{align*}
For this we calculate the gradient
\begin{align*}
    \nabla_{\vec{w}} \text{RSS}(\vec{w})
    &=\frac{1}{2}\Big(
        2\mat{X}^T\mat{X}\vec{w}+
        -2\mat{X}^T\vec{y}\Big)
\end{align*}
and then equals to cero to obtain that
\begin{align*}
    \mat{X}^T\mat{X}\vec{w} &= \mat{X}^T\vec{y}\\ 
    \Longleftrightarrow&\\
    \vec{w} &= \Big(\mat{X}^T\mat{X}\Big)^{-1}\mat{X}^T\vec{y}
\end{align*}
the hessian $H(\vec{w})=\mat{X}^T\mat{X}$ is always positive definite since $\vec{v}^T\mat{X}^T\mat{X}\vec{v}=||\mat{X}\vec{v}||^2\ge0$, therefore the function is convex and this is a minimum for the $\text{RSS}$.

\subsection{Maximum Likelihood Estimation}
For the MLE estimator we got that
\begin{align*}
    \vec{w}^* &=
    \argmin_{\vec{w}} \text{NLL}(\vec{w})\\
    &= \argmin_{\vec{w}} -\sum_{n=1}^N \p(y_n|\vec{x_n})\\
    &= \argmin_{\vec{w}} -\sum_{n=1}^N \log\p(y_n|\vec{x_n})\\
    &= \argmin_{\vec{w}} -\sum_{n=1}^N -\frac{1}{2\sigma^2}(y_n-\vec{w}^T\vec{x}_n)^2\\
    &= \argmin_{\vec{w}} \sum_{n=1}^N (y_n-\vec{w}^T\vec{x}_n)^2\\
\end{align*}
deriving w.r.t $\vec{w}$
\begin{align*}
    \nabla_{\vec{w}}\text{NLL}(\vec{w})
    &=\sum_{n=1}^N -2\big(y_n-\vec{w}^T\vec{x}_n\big)\vec{x}_n\\
\end{align*}
And finally when gradient is equal to zero
\begin{align*}
    \sum_{n=1}^N -2y_n\vec{x}_n + 2(\vec{w}^T\vec{x}_n)\vec{x}_n &= 0\\
    \Longleftrightarrow & \\
    \mat{X}^T\mat{X}\vec{w} &= \mat{X}^T\vec{y}\\
\end{align*}
which yields the same solution than OLS.

\subsection{Ridge regression}
When variables in linear regression are highly correlated, there will be ill-conditioned matrix which yields problems with inverse. This can be solved by using Ridge regression which suppose a normal prior with zero mean over parameters
\begin{equation*}
    \p(\vec{\theta}) = \normal(\vec{0},\frac{1}{\lambda}\mat{I})
\end{equation*}
pulling parameters to have lower values than vanilla linear regression but also solving ill-conditioned matrix. In matrix form this is equivalent to calculating the MAP estimation
\begin{align*}
    \vec{w}^*
    &= \argmax_{\vec{w}} \p(\vec{\theta}|\D)\\
    &= \argmax_{\vec{w}} \p(\D|\vec{\theta}) \p(\vec{\theta})\\
    &= \argmin_{\vec{w}} (\vec{y}-\mat{X}\vec{w})^T(\vec{y}-\mat{X}\vec{w}) + \lambda||\vec{w}||_2^2
\end{align*}
The solution to this problems is obtained in similar manner to OLS and the final result is
\begin{equation*}
    \vec{w}_{\text{map}} = \Big(\mat{X}^T\mat{X} + \lambda\mat{I}\Big)^{-1}\mat{X}^T\vec{y}
\end{equation*}

\subsection{Lasso regression}
Lasso is a similar model to Ridge, except the prior is suppose Laplace over parameters
\begin{equation*}
    \p(\vec{\theta}) = \prod_{d=1}^D \frac{1}{2b}\exp\Big(\dfrac{-|w_d|}{b}\Big)
\end{equation*}
which is equivalent to solve
\begin{align*}
    \vec{w}^*
    &= \argmin_{\vec{w}} (\vec{y}-\mat{X}\vec{w})^T(\vec{y}-\mat{X}\vec{w}) + \lambda||\vec{w}||_1
\end{align*}
Since this is not everywhere differentiable, we must use subgradients
\begin{equation*}
    \frac{d}{dx}|x|=
\begin{cases}
        { -1 } & \text{if } x<0 \\
        [-1, 1] & \text{if } x=0 \\
        { +1 } & \text{if } x>0
    \end{cases}
\end{equation*}
Therefore
\begin{align*}
    \Omega(\vec{w})
    &= \frac{1}{2}(\vec{y}-\mat{X}\vec{w})^T(\vec{y}-\mat{X}\vec{w}) + \lambda||\vec{w}||_1\\
    &= \sum_{n=1}^N \frac{1}{2}\Big(y_n - \sum_{d=1}^D x_{nd}w_d\Big)^2 + \lambda \sum_{d=1}^D|w_d|\\
    \Longrightarrow&\\
    \dfrac{\partial\Omega}{\partial w_\delta}
    &= -\sum_{n=1}^N \Big(y_n - \sum_{d=1}^D x_{nd}w_d\Big)x_{n\delta} + \lambda \frac{\partial}{\partial w_\delta}|w_\delta|\\
    &= -\sum_{n=1}^N \Big(y_n x_{n\delta} - x_{n\delta}^2 w_\delta - \sum_{d\neq\delta} x_{nd}x_{n\delta}w_d\Big)+ \lambda \frac{\partial}{\partial w_\delta}|w_\delta|\\
\end{align*}
Setting the gradient equals zero

\begin{align*}
    0
    &= -\sum_{n=1}^N \Big(y_n x_{n\delta} - x_{n\delta}^2 w_\delta - \sum_{d\neq\delta} x_{nd}x_{n\delta}w_d\Big) - \lambda \frac{\partial}{\partial w_\delta}|w_\delta|\\
    \Longleftrightarrow&\\
    w_\delta\sum_{n=1}^N x_{n\delta}^2
    &= \sum_{n=1}^N x_{n\delta}\Big( y_n - \sum_{d\neq\delta} x_{nd}w_d\Big) - \lambda \frac{\partial}{\partial w_\delta}|w_\delta|\\
    \Longleftrightarrow&\\
    w_\delta
    &= \dfrac{\sum_{n=1}^N x_{n\delta}\Big( y_n - \sum_{d\neq\delta} x_{nd}w_d\Big) - \lambda \frac{\partial}{\partial w_\delta}|w_\delta|}{\sum_{n=1}^N x_{n\delta}^2}\\
    &= \dfrac{\vec{x}_{:,\delta}^T  \vec{r}_{-\delta} - \lambda \frac{\partial}{\partial w_\delta}|w_\delta|}{||\vec{x}_{:,\delta}||_2^2}\\
    &= \dfrac{c_\delta - \lambda \frac{\partial}{\partial w_\delta}|w_\delta|}{a_\delta}\\  
\end{align*}
We can see that $c_\delta$ is proportional to the sample correlation coefficient between $\delta$ feature and the residual error of the target variable w.r.t the prediction using all variables except $\delta$; for that reason this is used to determine if its a relevant feature. Depending on $c_\delta$ sign we can establish a positive, negative or null correlation between target and feature.
For this reason
\begin{equation*}
    \lambda \frac{\partial}{\partial w_\delta}|w_\delta| = 
    \begin{cases}
        -\lambda & \text{if } w_\delta \propto c_\delta < 0 \\
        [-\lambda, \lambda] & \text{if } w_\delta \propto c_\delta = 0 \\
        +\lambda & \text{if } w_\delta \propto c_\delta >0
    \end{cases}
\end{equation*}
Finally assuming feature irrelevant when $w_\delta\in[-\lambda,\lambda]$ we get
\begin{equation*}
    w_\delta=
    \begin{cases}
        \frac{c_\delta + \lambda}{a_\delta} & \text{if } w_\delta \propto c_\delta < -\delta \\
        0 & \text{if } w_\delta \propto c_\delta \in [-\delta,\delta] \\
        \frac{c_\delta - \lambda}{a_\delta} & \text{if } w_\delta \propto c_\delta > \delta
    \end{cases}
\end{equation*}

\subsection{Elastic Net}

Finally elastic net is a model that combines Ridge regression and Lasso regression optimizing the following objective
\begin{equation*}
    \Omega(\vec{w}) &= (\vec{y}-\mat{X}\vec{w})^T(\vec{y}-\mat{X}\vec{w}) + \lambda_1||\vec{w}||_1 + \lambda_2||\vec{w}||_2^2\\
\end{equation*}
this works as a feature selection and feature clustering, grouping similar variables. The model is usually trained using coordinated descending which optimize one feature at a time while keeping fixed the rest.
\end{document}

