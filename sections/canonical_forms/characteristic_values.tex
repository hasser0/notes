\documentclass[../main.tex]{subfiles}

\begin{document}
\subsection{Characteristic values}
Finding a simple matrix representation for a linear transformation $T:V\mapsto V$ is worth because help us to calculate more easily any evaluation. The simplest representation of a transformation is a diagonal matrix.
\begin{equation*}
    D = \begin{bmatrix}
        \lambda_1 & 0 & \dots & 0\\
        0 & \lambda_2 & \dots & 0\\
        \vdots & \vdots & \ddots & 0\\
        0 & 0 & \dots & \lambda_n\\
    \end{bmatrix}
\end{equation*}
then for the vectors in the basis this is simply $T\alpha_i = \lambda_i \alpha_i$. This motivates our interest in finding such values and vectors, if those exists, that satisfy the previous relation. Such scalar $\lambda_i$ is called \textbf{characteristic value}. The set of vector such that $T\alpha = \lambda_i \alpha$ is a vector space also called \textbf{characteristic space} of $\lambda_i$ and any vector with this property is called a \textbf{characteristic vector} asociated to the characteristic value $\lambda_i$.

Searching for this values is done calculating the roots of the \textbf{characteristic polynomial} $\det(A-\lambda I)$, where $A$ is the matrix of $T$ related to any of its basis. We can choose any basis because any pair of similar basis has the same characteristic polynomial and all matrices of $T$ corresponding any basis are similar. Let $A = PBP^{-1}$ then
\begin{align*}
    \det(B-\lambda I) 
    &= \det(PBP^{-1}-\lambda PIP^{-1})\\
    &= \det(PBP^{-1}-\lambda I)\\
    &= \det(A-\lambda I)\\
\end{align*}

For any matrix A the following are equivalent
\begin{enumerate}
    \item A is diagonalizable
    \item the characteristic polynomial for A is $\prod(x-\lambda_i)^{d_i}$ with $\dim W_i = d_i$
    \item $\dim V = \sum \dim W_i$
\end{enumerate}
so we can find out if a matrix is diagonalizable by calculating the dimension of the chararacteristic spaces.
\end{document}