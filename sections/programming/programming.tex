% !TEX root = ../../main.tex
\documentclass[../../main.tex]{subfiles}

\begin{document}

\subsection{Programming languages}
\begin{itemize}
    \item Lexical analysis: Compilation phase that converts sequence of symbols into tokens
    \begin{itemize}
        \item Remove spaces
        \item Remove comments
        \item Create symbol table
    \end{itemize}
    \item Sintax analysis: Compilation phase that ensures tokens are correctly order
    \begin{itemize}
        \item Create abstract syntax tree
    \end{itemize}
    \item Semantic analysis: Compilation phase that checks meaning and consistency of code
    \begin{itemize}
        \item Type checking
        \item Symbol resolution
    \end{itemize}
    \item Variable: Named reference to a memory space
    \item Expression: Sequence of symbols that returns a value
    \item Operator: Symbol used to perform operations on operands
    \item Scope: Part of the program where a variable is visible
    \item Variable declaration: Process of assigning a memory location to a reference
    \item Variable initialization: Process of writting a value to memory location
    \item Value types: Variables types that store reference to the actual data
    \item Reference types: Variable types that store the value directly
    \item Object: Structure that contain variables and methods
    \item Mutable object: An object that can change its values
    \item Immutable object: An object that remains the same during the program execution
    \item Static programming language: Programming language that check variable type at compile time
    \item Dynamic programming language: Programming language that check variable type at runtime
    \item Strongly/weakly typed programming language: Programming language type system
    restriction level, not binary classification but rather a whole spectrum
\end{itemize}

\end{document}
