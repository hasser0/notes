\documentclass[../main.tex]{subfiles}

\begin{document}

\subsection{Databases - ACID}
\begin{definition}[Transaction]
    Collection of queries treated as a unit of work.
    \begin{itemize}
        \item Usually for insert and modify data.
        \item For read only transaction you can keep consistency.
        \item Use keywords BEGIN, COMMIT and ROLLBACK.
    \end{itemize}
\end{definition}

\begin{definition}[Atomicity]
    All queries in a transaction must succeed, otherwise all prior successfull queries must be rollback.
\end{definition}

\begin{definition}[Isolation]
    When multiple transactions read and write at the same time, inconsistency in the share data may arise. Therefore, run transactions in isolation one from others is very important
    \begin{itemize}
        \item Dirty reads
        \item Non repeatable reads
        \item Phantom reads
        \item Lost updates
        \item Isolation levels
        \begin{itemize}
            \item Read uncommitted: any change from outside is visible to the transaction
            \item Read commited: only see committed changes
            \item Repeatable read: resources remain unchange while transaction is running
            \item Snapshot: only see committed changes up to the start of the transaction
            \item Serializable
        \end{itemize}
        \item Locks
        \begin{itemize}
            \item pessimistic(row level, table level, page level)
            \item optimistic(no locks)
            \item repeatable reads lock
        \end{itemize}
    \end{itemize}
\end{definition}



\subsection{Reliability, Scalability and Mantainability}
\begin{definition}[Reliability]
    Keep working with hardware, software and human errors. To implement is important to decouple and isolate components, seek faults over failure, make easy to do the right thing and discourage the wrong, make fast to roll back changes.
\end{definition}
\begin{definition}[Scalability]
    Reasonable ways to deal with grows. To implement reliability is important to describe with metrics.
\end{definition}
\begin{definition}[Maintainability]
    It is divided into three parts
    \begin{itemize}
        \item Operability: Make operations easy. To implement track problem causes, automate deployment, define process.
        \item Simplicity: Easy for new engineers to understand. To implement seek abstraction over hardcode.
        \item Evolvability: Make easy for engineers to make changes.
    \end{itemize}
\end{definition}

\end{document}
