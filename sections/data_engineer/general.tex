\documentclass[../main.tex]{subfiles}

\begin{document}

\subsection{Databases - ACID}
\begin{definition}[Transaction]
    Collection of queries treated as a unit of work.
    \begin{itemize}
        \item Usually for insert and modify data.
        \item For read only transaction you can keep consistency.
        \item Use keywords BEGIN, COMMIT and ROLLBACK.
    \end{itemize}
\end{definition}

\begin{definition}[Atomicity]
    All queries in a transaction must succeed, otherwise all prior successfull queries must be rollback.
\end{definition}

\begin{definition}[Consistency]
    Refers to different parts of the data retrieving different information.
    \begin{itemize}
        \item Consistency in data: Referential integrity and business logic
        \item Consistency in reads: Retrieve the latest change as soon as committed.
        \begin{itemize}
            \item Eventual consistency
            \item Strong consistency
        \end{itemize}
    \end{itemize}
\end{definition}

\begin{definition}[Isolation]
    When multiple transactions read and write at the same time, inconsistency in the share data may arise. Therefore, run transactions in isolation one from others is very important
    \begin{itemize}
        \item Isolation problems
        \begin{itemize}
            \item Dirty reads: read uncommitted data
            \item Non repeatable reads: read committed data, updated in the middle of transaction
            \item Phantom reads: read committed data, inserted in the middle of transaction
            \item Lost updates: two transactions updating at the exact same time
        \end{itemize}
        \item Isolation levels
        \begin{itemize}
            \item Read uncommitted: uncommitted changes from outside are visible to the transaction
            \item Read committed: committed changes from outside are visible to the transaction
            \item Repeatable read: make sure not changes are done
            \item Snapshot: committed changes up to the start of the transaction
            \item Serializable
        \end{itemize}
        \item Locks
        \begin{itemize}
            \item pessimistic(row level, table level, page level)
            \item optimistic(no locks)
            \item repeatable reads lock
        \end{itemize}
    \end{itemize}
\end{definition}


\begin{definition}[Durability]
    Changes made by committed transactions must be persisted in a durable non-volatile storage
    \begin{itemize}
        \item Write ahead log
        \item Asynchoronous snapshot
        \item Append only file
    \end{itemize}
\end{definition}



\subsection{Reliability, Scalability and Mantainability}
\begin{definition}[Reliability]
    Keep working with hardware, software and human errors. To implement is important to decouple and isolate components, seek faults over failure, make easy to do the right thing and discourage the wrong, make fast to roll back changes.
\end{definition}
\begin{definition}[Scalability]
    Reasonable ways to deal with grows. To implement reliability is important to describe with metrics.
\end{definition}
\begin{definition}[Maintainability]
    It is divided into three parts
    \begin{itemize}
        \item Operability: Make operations easy. To implement track problem causes, automate deployment, define process.
        \item Simplicity: Easy for new engineers to understand. To implement seek abstraction over hardcode.
        \item Evolvability: Make easy for engineers to make changes.
    \end{itemize}
\end{definition}

\end{document}
