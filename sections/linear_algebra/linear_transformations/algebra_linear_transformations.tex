% !TEX root = ../../main.tex
\documentclass[../main.tex]{subfiles}

\begin{document}
\subsection{Algebra of linear transformations}
Let $V$ and $W$ be vector spaces, we denote $L(V,W)$ as the set of linear transformations $T:V\mapsto W$. It's worth noting that $L(V,W)$ with addition and scalar multiplication defined as usual is a vector space. The dimension of such a vector space is
\begin{equation*}
    \dim L(V,W) = \dim V \cdot \dim W
\end{equation*}
if $\{\alpha_1,\dots,\alpha_n\}$ is a basis for V, $\{\beta_1,\dots,\beta_m\}$ is a basis for W and we define $nm$ linear transformations $E^{p,q}$ such that $E^{p,q}\alpha_i = \delta_{i,p}\beta_q$ it can be demonstrated that this form a basis for $L(V,W)$.

The following statements are easily demonstrated.
\begin{itemize}
    \item Composition of linear transformations is also a linear transformation.
    \item The inverse of a linear transformation (if exists) is a linear transformation.
    \item A linear operator is a linear transformation where the domain and codomain is the same. 
\end{itemize}
and all together tell us that $L(V,W)$ with addition, scalar multiplication and composition as product form a \textbf{linear algebra} which is a special kind of algebraic structure.

A non-singular transformation is such that $N_T=\{0\}$. A result tell us that a linear transformation is non-singular iff it maps linearly independent sets onto linearly independent sets. If $T:V\mapsto W$ and $\dim V = \dim W$ then
\begin{enumerate}
    \item $T$ is invertible
    \item $T$ is non-singular
    \item $T$ is onto 
    \item $\{\alpha_1,\dots,\alpha_n\}$ basis of $V$ then $\{T\alpha_1,\dots,T\alpha_n\}$ basis of $W$
\end{enumerate}
are equivalent.
\end{document}
