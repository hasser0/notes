% !TEX root = ../../main.tex
\documentclass[../main.tex]{subfiles}

\begin{document}
\subsection{Matrix representation of linear transformations}
Linear transformations can be represented as a matrix. For example, let $\mathfrak{B}=\{\alpha_1,\dots,\alpha_n\}$ a basis for $V$, $\mathfrak{B}'=\{\beta_1,\dots,\beta_m\}$ a basis for $W$ and $T:V\mapsto W$ such that $T\alpha_j=\sum_{i=1}^m A_{ij}\beta_i$. If $\alpha=\sum_{j=1}^n x_j\alpha_j$, then we got
\begin{align*}
    T\alpha = T\Big(\sum_{j=1}^n x_j \alpha_j\Big)
    &=\sum_{j=1}^n x_j T\alpha_j\\
    &=\sum_{j=1}^n x_j \sum_{i=1}^m A_{ij}\beta_i\\
    &=\sum_{j=1}^n \sum_{i=1}^m A_{ij}x_j\beta_i\\
    &=\sum_{i=1}^m \Big(\sum_{j=1}^n A_{ij}x_j\Big)\beta_i\\
    &\Longleftrightarrow\\
    \big[T\alpha\big]_\mathfrak{B'}&=
    A\big[\alpha\big]_\mathfrak{B}
\end{align*}
where A is called the matrix of $T$ relative to basis $\mathfrak{B}$ and $\mathfrak{B'}$. 

Since $A_{ij}$ are unique for each pair of basis and each transformations, it may be reasonable to think about a function $\gamma:L(V,W)\mapsto F^{n\times m}$. In fact this function is linear, so is a linear transformation; suppose $\gamma(T) = A$ and $\gamma(U) = B$, now in a similar way
\begin{align*}
    (cT+U)\alpha &= (cT+U)\Big(\sum_{j=1}^n x_j \alpha_j\Big)\\
    &=cT\Big(\sum_{j=1}^n x_j \alpha_j\Big) + U\Big(\sum_{j=1}^n x_j \alpha_j\Big)\\
    &=c\sum_{i=1}^m\sum_{j=1}^n A_{ij}x_j\beta_i + \sum_{i=1}^m \sum_{j=1}^n B_{ij}x_j\beta_i\\
    &=\sum_{i=1}^m\Big(\sum_{j=1}^n (cA_{ij} + B_{ij})x_j\Big)\beta_i\\
    &\Longleftrightarrow\\
    \gamma(cT+U)&=c\gamma(T)+\gamma(U)
\end{align*}
Because $\gamma$ is a linear function one-one and $\dim L(V,W) = \dim F^{n\times m}$ this means $\gamma$ is an isomorphism of V and W, equivalent to saying that $L(V,W)$ and $F^{n\times m}$ are isomorphic.

Composition of linear transformations $(UT)$ can be represented as a product of matrices. If $A$ is the matrix of T relative to basis $\mathfrak{B}$ and $\mathfrak{B'}$ and $B$ is the matrix of U relative to basis $\mathfrak{B'}$ and $\mathfrak{B''}$ then
\begin{equation*}
    \big[(UT)\alpha\big]_\mathfrak{B''}=
    BA\big[\alpha\big]_\mathfrak{B}
\end{equation*}
this can be proofed using a similar argument to the previous ones.

Note that the matrix $M_{\mathfrak{B}\mapsto\mathfrak{B'}}$ corresponding to the basis $\mathfrak{B}$ and $\mathfrak{B'}$ for the transformation M which carries vectors from $\mathfrak{B}$ to $\mathfrak{B'}$ is called the change of basis matrix from $\mathfrak{B}$ to $\mathfrak{B'}$. This matrix has the property that 
\begin{align*}
    [\alpha]_{\mathfrak{B'}}&=
    M_{\mathfrak{B}\mapsto\mathfrak{B'}}[\alpha]_{\mathfrak{B}}
\end{align*}

Now, we treat cases when $T$ is a linear operator and $[T]_{\mathfrak{B}}$ is the matrix of $T$ corresponding to $\mathfrak{B}$.
\begin{alignat*}{3}
    &
    &\big[T\alpha\big]_{\mathfrak{B}} 
    &=\big[T\big]_{\mathfrak{B}}\big[\alpha\big]_{\mathfrak{B}}
    \\
    \Longleftrightarrow&
    &\big[T\alpha\big]_{\mathfrak{B}} 
    &=\big[T\big]_{\mathfrak{B}}M_{\mathfrak{B'}\mapsto\mathfrak{B}}[\alpha]_{\mathfrak{B'}}
    \\
    \Longleftrightarrow&
    &M_{\mathfrak{B}\mapsto\mathfrak{B'}}\big[T\alpha\big]_{\mathfrak{B}} 
    &=M_{\mathfrak{B}\mapsto\mathfrak{B'}}\big[T\big]_{\mathfrak{B}}M_{\mathfrak{B'}\mapsto\mathfrak{B}}[\alpha]_{\mathfrak{B'}}
    \\
    \Longleftrightarrow&
    &\big[T\alpha\big]_{\mathfrak{B'}} 
    &=M_{\mathfrak{B}\mapsto\mathfrak{B'}}\big[T\big]_{\mathfrak{B}}M_{\mathfrak{B'}\mapsto\mathfrak{B}}[\alpha]_{\mathfrak{B'}}
    \\
    \Longleftrightarrow&
    &\big[T\big]_{\mathfrak{B'}} 
    &=M_{\mathfrak{B}\mapsto\mathfrak{B'}}\big[T\big]_{\mathfrak{B}}M_{\mathfrak{B'}\mapsto\mathfrak{B}}
    \\
\end{alignat*}
Any pair of square matrices $A,B\in F^{n\times n}$ are said to be similar if exists $P\in F^{n\times n}$ invertible such that
\begin{equation*}
    B = PAP^{-1}
\end{equation*}
\end{document}

