\documentclass[../main.tex]{subfiles}

\begin{document}


\subsection{Formal languages}

\begin{definition}[Formal language]
    A formal language is described by
    \begin{enumerate}
        \item set of symbols
        \item rules to form wffs
        \item translations between english and the formal language
    \end{enumerate}
\end{definition}


\subsection{Expressions and well formed formulas}


\begin{definition}[Symbols]
    There are two types of symbols
    \begin{enumerate}
        \item logical: meaning does not change
        \begin{itemize}
            \item connective
            \item punctuation
        \end{itemize}
        \item non logical: meaning is not fixed
        \begin{itemize}
            \item sentences
        \end{itemize}
    \end{enumerate}
\end{definition}
\begin{definition}[Expression]
    An expression is an ordered sequence of symbols
\end{definition}
\begin{definition}[Recursive definition wffs]
    A wff meets one of the following conditions:
    \begin{enumerate}
        \item sentences symbols are wffs
        \item $\neg\alpha$, $\alpha\wedge\beta$, $\alpha\vee\beta$, $\alpha\leftrightarrow\beta$, $\alpha\rightarrow\beta$ are wffs whenever $\alpha, \beta$ are wffs
    \end{enumerate}
\end{definition}
\begin{definition}[Construction sequence definition wffs]
    A wff is the last expression of a sequence of expressions $<\varepsilon_1,\dots,\varepsilon_n>$ where
    \begin{enumerate}
        \item $\varepsilon_i$ is a sentence
        \item $\varepsilon_i = E_\neg(\varepsilon_j)$ where $i>j$
        \item $\varepsilon_i = E_\blacksquare(\varepsilon_j, \varepsilon_k)$ where $i>j,k$ and $\blacksquare\in \{ \wedge, \vee, \rightarrow, \leftrightarrow \}$
    \end{enumerate}
\end{definition}


\subsection{Truth assignments}


\begin{definition}[Truth assignment]
    A truth assignment $v$ for a set $S$ of sentences is a function $S\rightarrow\{T,F\}$
\end{definition}
\begin{definition}[Truth assignment extension]
    The extension $\bar{v}$ of a truth assignment $v$ is a function $\bar{S}\rightarrow\{T,F\}$ ($\bar{S}$ the set of wff built from $S$) such that
    \begin{enumerate}
        \item $\bar{v}(A) = v(A)$ for $A$ sentence
        \item $\bar{v}(A\wedge B)$, $\bar{v}(A\vee B)$, $\bar{v}(A\leftrightarrow B)$, $\bar{v}(A\rightarrow B)$ meets their truth tables
    \end{enumerate}
\end{definition}
\begin{definition}[Truth assignment satisfies a wff]
    A truth assignment satisfies a wff $\phi$ iff $\bar{v}(\phi) = T$
\end{definition}
\begin{definition}[Satisfiable set of wffs]
    A set of wffs is satisfiable iff exists truth assignment that satisfies all wffs of the set
\end{definition}
\begin{definition}[Tautologically implies]
    A set of wff $\Sigma$ tautologically implies $\tau$ written as $\Sigma\models\tau$ iff every truth assignment $v:S_{\Sigma, \tau} \rightarrow \{T,F\}$ that satisfies $\Sigma$ also satisfies $\tau$ where $S_{\Sigma, \tau}$ is the set of sentences in $\Sigma$ and $\tau$
\end{definition}
\begin{definition}[Tautologically equivalent]
    It's said that $\sigma$ is tautologically equivalent to $\tau$ iff $\sigma\models\tau$ and $\tau\models\sigma$
\end{definition}
\begin{definition}[Tautology]
    It's said that $\tau$ is a tautology iff $\emptyset\models\tau$
\end{definition}
\begin{definition}[Equivalent sets of wffs]
    Two sets of wffs $\Sigma, \Pi$ are equivalent iff are tautologically equivalent.
\end{definition}
\begin{definition}[Independent set of wffs]
    A sets of wffs is independent iff no member of it is tautologically implied by the others 
\end{definition}


\subsection{Induction and recursion}


\begin{definition}[Closed set under set of functions]
    $S$ is closed under $\mathcal{F}$ iff $\alpha_1,\dots,\alpha_n\in S$, $f\in\mathcal{F}$ then $f(\alpha_1,\dots,\alpha_n)\in S$
\end{definition}
\begin{definition}[Inductive set on a basis under a set of functions]
    $S$ is inductive on $B$ under $\mathcal{F}$ iff  $B\subseteq S$ and $S$ is closed under $\mathcal{F}$
\end{definition}
\begin{definition}[Freely generated set]
    $C$ is freely generated from $B$ by $\mathcal{F}$ iff $\forall f\in\mathcal{F}$ $f$ is bijective and $\forall f,g\in \mathcal{F}$ $\mathcal{R}_{f|_C}, \mathcal{R}_{g|_C}, B$ are pairwise disjoint
\end{definition}
\begin{definition}[Generated set from a basis under a set of functions]
    $C$ is generated from $B$ under $\mathcal{F}$ iff the following equivalent are true
    \begin{enumerate}
        \item $C=\bigcup_n C_n$ where $C_n$ is the set of expressions of lenght $n$ built from $B$ under $\mathcal{F}$
        \item $C=\bigcap S$ where $S$ is inductive on $B$ under $\mathcal{F}$
    \end{enumerate}
\end{definition}
\begin{theorem}[Induction principle]
    Let $C$ generated from $B$ by $\mathcal{F}$, then if $B\subseteq S \subseteq C$ and $S$ is closed under $\mathcal{F}$ then $S=C$.
\end{theorem}
\begin{corollary}[Induction principle for wffs]
    If a set of wff $S$ contains any sentence and is closed under sentential connectives then $S$ is the set of wff.
\end{corollary}
\begin{theorem}[Recursion theorem]
    If $C$ is freely generated from $B$ by $\mathcal{F}$, $h:B\rightarrow V$ and $t:\mathcal{F}\rightarrow\mathcal{G}$ then exists unique function $\bar{h}: C\rightarrow V$ such that
    \begin{enumerate}
        \item $\bar{h}|_B = h|_B$
        \item $\bar{h}(f(x_1,\dots,x_n)) = t(f)(\bar{h}(x_1),\dots,\bar{h}(x_n))$
    \end{enumerate}
\end{theorem}


\subsection{Boolean functions}


\begin{definition}[k-boolean function]
    A k-boolean function is a function $\mathcal{B}:\{T,F\}^k\rightarrow\{T,F\}$
\end{definition}
\begin{theorem}
    Every wff $\alpha$ defines a boolean function $\mathcal{B}_\alpha$
\end{theorem}
\begin{theorem}
    $\alpha\models\beta$ iff $\mathcal{B}_\alpha(X)\le\mathcal{B}_\beta(X)$
\end{theorem}
\begin{theorem}
    $\models\alpha$ iff $\mathcal{B}_\alpha(X)=T$
\end{theorem}
\begin{definition}[Disjunctive normal form]
    A wff $\alpha$ is in disjunctive normal form iff
    $$
    \alpha=\gamma_1\vee\cdots\vee\gamma_n
    $$
    where each $\gamma_i=\beta_{i1}\wedge\cdots\wedge\beta_{ik}$ and $\beta_{ij}$ is a sentence or negative of sentence
\end{definition}
\begin{definition}[Conjunctive normal form]
    A wff $\alpha$ is in conjunctive normal form iff
    $$
    \alpha=\gamma_1\wedge\cdots\wedge\gamma_n
    $$
    where each $\gamma_i=\beta_{i1}\vee\cdots\vee\beta_{ik}$ and $\beta_{ij}$ is a sentence or negative of sentence
\end{definition}
\begin{theorem}
    Every boolean function has an equivalent wff that can be found constructing the disjunctive normal form from the boolean function truth table
\end{theorem}
\begin{definition}[Complete set of connectives]
    A set of connectives is complete iff any boolean function is equivalent to a wff using only those connectives
\end{definition}
\begin{definition}[Substitution]
    $\phi_\blacksquare^A$ is defined as the resulting wff from substituting $A$ by $\blacksquare$
\end{definition}
\begin{definition}[Resolution]
    $\phi_*^A = \phi_\top^A \vee \phi_\bot^A$
\end{definition}
\begin{theorem}
    The following three statements are true
    \begin{itemize}
        \item $\phi\models\phi_*^A$
        \item if $\phi\models\varphi$ and $A$ does not appear in $\varphi$ then $\phi_*^A\models\varphi$ is satisfiable
        \item $\phi$ is satisfiable iff $\phi_*^A$ is satisfiable
    \end{itemize}
\end{theorem}
\end{document}